\chapter{Conclusion and Future Work}
We try to advance the search method for math expression using sub-paths.
This is a structure-based approach (as described in section~\ref{Structure-based}) to search and assess mathematical expression similarity.
Without using vector-space model as opposed to its presence on text-based approach, our method has tried to further develop structure-based method and offer some new insights for future research, 
and the evaluation results already indicate the feasibility of our methods.
One of our hypothesis is structure-based method better captures mathematical expression semantics. 
We believe there is large potential to further refine our novel similarity-search method as our presented attempt to improve mathematical-similarity search has demonstrated some promising aspects of our method, and lured some new ideas to improve our method.

\section{Summary and Conclusion}
Our method tries to measure math-expression similarity from the perspective of their structures and operand symbols. 
We search structurally relevant expressions in a subset of index and our evaluation results have demonstrated the overall cost-effectiveness of adding symbolic similarity-search method on proposed structural similarity-search.
This paper's main contribution includes, 
first gives definitions and algorithms for evaluating the similarity between two math expressions to allow a math-aware search system to rank candidate expressions against a query expression;
second gives a novel search method to reduce the search set of structurally similar expressions given a query. 
By implementing those ideas and conducting experiments on our dataset, we have offered insights into a new way to perform matching of structured content in general.
Given the newness of the domain, we have showed an appealing new direction to handle math expression search from an angel of both symbolic and structural similarity. 

\section{Current Problems}
We expect our index to be less redundant because our method does not need augmentation to capture the same semantical meanings between different alternative forms of a math expression,
while it is observed that our index size is still relatively large (9.4GB) compared to the size of input raw data (around 60MB).
There are presumably two reasons for this: 
Firstly we are implementing our index based on file system directories for the ease of implementation, the overhead for constructing directories in file system cannot be overlooked. 
Secondly, the 60MB raw data are all math formula in \LaTeX,	in contrast to a general text search engine, whose raw data usually contains the entire document text, our raw data is comparatively more data-intensive, and carries more information than general text. 
Moreover, we do not use any kind of compression in our index.
Another observation is, the absolute value of our query lookup time is not ideal, especially
the first-time querying takes much longer time (because cache miss) than the time of subsequent same queries. 
Furthermore, queries with simple decomposed sub-paths takes relative much longer time to search (e.g. query number 2 takes almost 4 seconds in average). 

\section{Future Work}
In the next stage, trying to reduce the overhead for searching the index can be a good way to fix our bottleneck on both improving storage space efficiency and speeding up query look-up process. 
We may try to index the data into a single large file (e.g. implement a B-tree like file structure to store sub-paths) instead of small files spread over directories, to lower the system overhead in storage and take advantage of cache locality. 
It maybe also a good idea to try compressing the index and saving less information on disk for a branch word to further reduce the storage demand for our index.
Nevertheless, Considering the fact that our system is newly created, we have already met the goal to test our new ideas on concerned domain, despite the less usability compared to already mature existing tools for general text search.

Also, the experiments we have is not based on popular NTCIR dataset in this research domain, because of the problem we mentioned in section~\ref{datasetAndIndex}. 
This leaves a comparison of our method with mainstream math information retrieval methods to be desired.  
In the future, we may try to create a new parser for MathML content so that we are able to give experiments on some standard dataset of math IR, with larger data size and more testing queries, in order to provide more comparable results on evaluation.

Other than the problems addressed above, it is also desired to integrate general text search ability into our math-only search method. 
Additionally, the manner we use to break math formula into branch words and to index them through posting list makes it easy to parallelize and distribute the searching process,
which means there is a large potential for future efforts to improve the efficiency of this method.
