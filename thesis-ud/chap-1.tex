\chapter{Background}
Apart from general text content, structured information is also widely contained by digital document. Among these, a lot of mathematical content (including documents on Internet), are represented using markups like \LaTeX\  or MathML~\footnote{\url{http://www.w3.org/Math/}}, which is in a rich structural way. 
Information Retrieval on those structured data in mathematics language is not that well-studied or exhaustively covered by mainstream IR research, compared to that with general text. 
Thus it can be challenging yet very helpful given the contribution and importance of mathematics to our science. 

However, the structured sense of mathematical language, as well as many its semantic properties (see section~\ref{measure_sim}), makes general text retrieval models deficient to provide good search results. Through this paper, we have made our efforts to tackle some of these problems. 
Some of the ideas used in this paper deals with "tree structured" data in a general way, have the potential to be applied by other fields of structured data retrieval besides that from mathematical language. 

\section{Math IR Categories}

Mathematical information involves a wide spectrum of topics, 
we are, of cause, not focusing on every aspects in mathematical information retrieval. 
It is good to clarify our concentration in this paper here, by first listing a set of categories that a mathematical information retrieval system may be classified into,
and define our target field of study.

\pagebreak
Listed here, are considered four possible categories of topic for mathematical information retrieval:

\begin{enumerate}
\item Boolean or Similarity Search
\item Math Detection and Recognition
\item Evaluation, Derivation and Calculation
\item Other topics 
\end{enumerate}

The first one is doing mathematical information retrieval by searching, and finding the most relevant context of document that matches the query, very similar to the most common ways that other general text search engines will do, by boolean or similarity search~\cite{iir}. 
The only difference is, the query may contain mathematical expressions. 
Instance of such search engine can be useful in many ways, for example, student may utilize it to know which identity can be applied to a formulae in order to give a proof of that formulae.
This is the area where we focus in this paper. Specifically, we are proposing a series of methods for similarity search of math content. And our method is using query only in \LaTeX\ markup 
(some math-aware search engines~\footnote{WolframAlpha: \url{https://www.wolframalpha.com/} and Zentralblatt math \\from MathWebSearch: \url{http://search.mathweb.org/zbl/}} support queries in mathematical formulae and normal text together), and return documents ordered by score which indicates the similarity degree. 

Digital mathematical content document can also be in an image format (e.g. generated by a handwritten query), thus to retrieve these information involves detection or recognition. Inspired by the advances from deep learning, we may foresee a large potential to be explored on topics related to this. 

Because the nature of mathematical language, a query (e.g. an algebra expression) can be evaluated and potentially derived into an alternate form, or calculated. 
The result value of evaluation or derived form may also be considered being relevant to that query. 
These potentially require a system to handle symbolic or value calculation, or even a good knowledge of derivation rules implied by different mathematical expression
(e.g. computational engine \textit{Symbolab}~\footnote{Symbolab Web Search: \url{http://www.symbolab.com}}).

Besides the first three categories, there are many other topics. Knowledge mining, for example, will need deeper level of understanding on math content. Topics like this need system to answer a query like: ``Find an article related to the \textit{Four Color Theorem}", which is an example from NTCIR Math task open information retrieval~\cite{ntcirtopic}.

\section{Issues in Measuring Similarity}
\label{measure_sim}

For the first one we would write extended expand.


Some publications already cover a lot on the difficulties on mathematical content information retrieval, also in respective of the differences between mainstream IR approach with those in math search, and the inadequacy to apply general text search method (e.g. bag of words model) to content in mathematical language.

well listed in paper [x]

Search for mathematical expressions is difficult for a number of reasons.

\section{Related Work}

Our system Cowpie \footnote{demo page: \url{infolab.ece.udel.edu:8912/cowpie/}} \cite{WolframAlpha}

Search systems play an essential role in our daily lives because there are
a huge number of pages on the Web. However, we sometimes encounter
problems with how to search for or what to submit as a query when we
want to search for things that cannot be easily expressed in natural language.

Information Retrieval technology has reached maturity, math retrieval is still in
its nascent stages, and many challenges remain. Those challenges are due in part
A math expression is often written in a symbolic language with several lev-
els of abstraction, and often contains rich structural information. Additionally,
notational ambiguities, and syntactical and semantic equivalences, make math
knowledge harder to search.


As computers appeared, one of the important tasks that emerged was
storing and searching large sets of documents. The inverted index was an
early and natural development in this field. It can be thought of like an index
in a book: it is a structure that maps each term to a list of documents that
contains it. In its most simple form, an inverted index only allows us to find
the documents that contain a given term

More and more math knowledge has become available on the Web, and search
is a gate to such vast treasure of digital mathematics content [11]. Even though
Information Retrieval technology has reached maturity, math retrieval is still in
its nascent stages, and many chal

Unlike general text content, mathematical language, by its nature, has many differences from other textual documents. 

Math information retrieval (MIR) starts to be recognized
as an important very domain-specific sort of information re-
trieval research field.

Mathematical expressions are objects with complex
structures and rather few distinct symbols and terms.

many challenges remain. Those challenges are due in part
to the significant difference of math knowledge from other.
A math expression is often written in a symbolic language with several lev-
els of abstraction, and often contains rich structural information.

\section{Overview}
\subsection{Considerations}
\subsection{Property}

interchangeably 

MathML vs LaTeX

First,
one mathematical formula may use different notations and
thus has different representations. 1 Second, two semanti-
cally different mathematical document can contain the same
number of terms. 2 Also, the order of terms in math language
sometimes matters but can be mutable in other cases, which
implies it is impractical to uniformly apply one single lan-
guage model to mathematical content.

\section{Related Work}

An inverted index is the tool used most commonly in text search. It is
a lookup table from words to the documents that contain them. This idea has
been modified to allow for word locality, phrase searching, index compression,
and distributed indexing. With these and other improvements, systems can
be built to quickly search massive
