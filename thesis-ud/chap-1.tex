\chapter{Introduction}
Apart from general text content, structured information is also widely contained by digital document. 
Among these, a lot of mathematical content are represented, they are primarily using \LaTeX, which is a rich structural markup language. 
Information Retrieval on those structured data in mathematics language is not as well-studied or exhaustively covered as that is in general text IR research. 
To better search mathematical content can be significantly meaningful in terms of extending our border on information retrieval.


However, the structured sense of mathematical language, as well as many its semantic properties (see section~\ref{measure_sim}), makes general text retrieval models deficient to provide good search results. 
This is because some fundamental differences between mathematical language and general text.
Through this paper, we have made our efforts to tackle some of the problems we are having to search mathematical language. 
Some of the ideas used in this paper that deals with "tree structured" data can be generalized and potentially applied to other fields of structured data retrieval. 

\section{Math IR Domains}

Mathematical information involves a wide spectrum of topics, 
\cite{goodsurvey} gives a comprehensive review on mathematical IR researches.
We are of cause not focusing on every aspect in mathematical information retrieval. 
It is good to clarify our concentration in this paper here by first listing a set of concentrations in the related research area and define our target field of study.

\pagebreak
Listed here, are considered three major topics for mathematical information retrieval:

\begin{enumerate}
\item Boolean or Similarity Search
\item Math Detection and Recognition
\item Evaluation, Derivation and Calculation
\end{enumerate}

Boolean/Similarity search finds or ranks mathematical expressions against a query. 
They define the criteria of matching expressions or dimensions of similarity between two expressions.
This is analogy to the boolean or similarity search of general text search engines,
except the query and document may contain mathematical expressions. 
Some search engines deal with only formula (e.g. 
SearchOnMath~\footnote{\url{http://searchonmath.com}},
Uniquation~\footnote{\url{http://uniquation.com/en}}
and 
Tangent~\footnote{\url{http://saskatoon.cs.rit.edu/tangent/random}}
) 
and some math-aware search engines (e.g.
WolframAlpha~\footnote{\url{https://www.wolframalpha.com/}} and 
Zentralblatt math from MathWebSearch~\footnote{\url{http://search.mathweb.org/zbl/}}
)
are able to search query with mixed text and mathematical formula.
These search engines can be useful in many ways, for example, student may utilize it to know which identity can be applied to a formulae in order to give a proof of that formulae.
This is the area where we focus in this paper. 

Digital mathematical content document can also be in an image format (e.g. a handwritten formula), topics on retrieving information in these image requires detection and recognition of their visual features (texture, outline, shape etc.).

Also, because the nature of mathematical language, a query (e.g. an algebra expression) can potentially derived into alternate forms, or calculated and evaluated into a value. 
These potentially require a system to handle symbolic or numeric calculation, or even a good knowledge of derivation rules implied by different mathematical expression. 
Those numeric search engines
(e.g. computational engine Symbolab~\footnote{Symbolab Web Search: \url{http://www.symbolab.com}} and WolframAlpha)
can help evaluate mathematical expressions and reveal the deeper information contained from those expressions.

Besides the three concentrations mentioned, there are many other topics. Knowledge mining, for example, will need deeper level of understanding on math language. A typical goal of this topic is to give a solution or answer based on all the document information retrieved. e.g. ``Find an article related to the \textit{Four Color Theorem}"~\cite{ntcirtopic}.

These topics somehow overlap in some cases, for example, some derivation can be used to better assess the similarity between math formulae, e.g. $\dfrac{a + b}{c}$ and $\dfrac{a}{c} + \dfrac{b}{c}$ should be considered as relevant because their equivalent form of representation.
Similarly, mathematical knowledge is required to understand the same meaning (thus high similarity) between $ \dbinom n 1 $ and $C_n^1$.
So boolean or similarity search possibly involves a level of understanding on mathematical language. 
However, we are not going to include these problems into our research domain, instead, this paper addresses mathematical expression similarity from only structural and symbolic perspectives.

\section{Issues in Measuring Similarity}
\label{measure_sim}
Unlike general text content, mathematical language, by its nature, has many differences from other textual documents, there are a number of new problems in measuring mathematical expression similarity. 
Even without caring about the possible derivations and high level knowledge inference, there are still a set of new problems for measuring mathematical similarity.

Firstly, the differences between mathematical expressions should be captured in a cooperative manner in order to measure similarity,
because only respecting symbolic information is not sufficient in mathematical language.
To illustrate this point, we know that
$ax+(b+c)$ in most cases is not equivalent to $(a+b)x+c$ although they have the same set of symbols, because the different structure represents a different semantic meaning.
And the order of tokens in math expression can be commutative in some cases but not always, for example, commutative property in math makes $a+b=b+c$ for addition operation, but on the other hand $\dfrac a b$ is most likely not equivalent to $\dfrac b a$.
These characteristics make many general text search methods (e.g. \textit{bag of words} model, \textit{tf-idf} weighting) inadequate.  
Moreover, symbols can be used interchangeably to represent the same meaning, e.g. $a^2+b^2=c^2$ and $x^2+y^2=z^2$. 
However, interchanging symbols does not always preserve mathematical semantics, changes of symbols in expression preserve more syntactic similarity when changes are made by substitution, e.g. Given query $x(1+x)$, expression $a(1+a)$ are considered more relevant than $a(1+b)$. 

Secondly, how we evaluate structural similarity between expressions is a question. A complete query may structurally be a part of a document, or only some parts of a query match somewhere in a document expression.
In cases when a set of matches occur within some measure of ``distance", we may consider them to contribute similarity as a whole, but when matches occur ``far away" for a query expression, then under the semantic implication of mathematics, they probably will not contribute the similarity degree in any way.
We need metrics to score these similarity under certain rules for relevance assessments. 

Lastly, when trying to capture more semantic information from expressions, we can improve our measurement on similarity but it may introduce more ambiguity. 
For example, semantically incorrect math markups in document, e.g. using ``sin" in \LaTeX\ markup instead of macro ``\textbackslash sin", will make it difficult to tell whether it is a product of three variables or a \textit{sine} function if we want to capture their semantical meaning in such a depth. 
And depending on what level of semantical meaning we want to capture, ambiguity cases can be different. 
Consider $f(2x+1)$, if we want to know if $f$ is a function rather than a variable, the only possibility is looking for its contexts, but we can nevertheless always think of it as a product without losing the possibility to search similar expression like $f(1 + 2y)$, the same way goes reciprocal $a^{-1}$ and inverse function $f^{-1}$. 
Most often, even if no semantic ambiguity occurs, efforts are needed to capture some semantical meanings. e.g. In $\sin 2 \pi$, it is not easy to figure out,
 without a knowledge on trigonometric function convention, that $2 \pi$ is the subordinative of sine function.

\section{Contribution Summary}
This paper has proposed new ideas to search math expressions in a limited search set and evaluate both structural and symbolic similarity of two two math expressions.
Based on an existing idea of using leaf-root path (which will be described at section~\ref{leafrootmethod}) to search mathematical expressions, we have further developed it to index structural information for easy discover of structurally similar document expressions. 
To summarize, the research presented contributes the following points to the field of mathematical information retrieval.

\begin{itemize}
\item The method based on operation tree representation of mathematical expression to index information by sub-paths and search math query through simultaneously merging sub-paths. This also enables the ability to prune search set and potentially provide ways to parallelize search process. 
\item Formally defined sub-expression relation between two mathematical expressions. Based on this definition, we have observed some interesting properties, and depending on this, we have proposed a new search method which can limit the indexed expressions being possible sub-expression of query to a subset of index.
\item An algorithm to ``check'' defined sub-expression relation between two mathematical expressions. 
\item An algorithm to evaluate symbolic similarity between two mathematical expressions with consideration of symbol $\alpha$-equality. 
\item Propose a ranking schema to combine three factors (matching ratio, matching depth, and symbolic similarity score) as a tuple to score overall similarity between a query and a document math expression.
\item A parser to directly parse \LaTeX\ math mode content into in-memory operation tree representation, omitting semantic-irrelevant content. 
\item A set of tools that implement our idea with source code avaiable online: \url{https://bitbucket.org/t-k-/cowpie}.
And a newly-assembled corpus from StackExchange to provide a \LaTeX\ (instead of MathML) content dataset for public assessment and evaluation. 
\end{itemize}
